\noindent\begin{tabularx}{\linewidth}{|c|X|X|X|}
\hline
& \multicolumn{1}{c|}{\textbf{1. 代数的手続きの理解}} & \multicolumn{1}{c|}{\textbf{2. 関数的思考の理解}} & \multicolumn{1}{c|}{\textbf{3. 直交座標上での表現の理解}}\\\hline
\scriptsize(1) & {\cellcolor{c011}\scriptsize
\renewcommand{\arraystretch}{1.6}
\begin{tabularx}{\linewidth}{X}
    \mit $y$ に $6$ を代入し,$x$ に関する\textbf{方程式} $\mathbf{6}=\dfrac{\mathbf{1}}{\mathbf{2}}\bm{x}+\mathbf{2}$ を解く。
\end{tabularx}\renewcommand{\arraystretch}{1}}
 & \cellcolor{mygray}
 & \cellcolor{mygray}
\\\hline
\scriptsize(2) & {\cellcolor{c021}\scriptsize
\renewcommand{\arraystretch}{1.6}
\begin{tabularx}{\linewidth}{X}
    \mit $y=ax+b$ に $x=-3$,$y=5$ を代入した $\mathbf{5=-3}\bm{a}+\bm{b}$ と,$x=12$,$y=0$ を代入した $\mathbf{0=12}\bm{a}+\bm{b}$ の\textbf{連立方程式}を解く。
\end{tabularx}\renewcommand{\arraystretch}{1}}
 & \cellcolor{mygray}
 & \cellcolor{mygray}
\\\hline
\scriptsize(3) & \cellcolor{mygray}
 & {\cellcolor{c032}\scriptsize
\renewcommand{\arraystretch}{1.6}
\begin{tabularx}{\linewidth}{X}
    \mit $\bm{y}$ \textbf{が} $\bm{x}$ \textbf{の関数である}とき,$\bm{x}$ \textbf{の値を決めると} $\bm{y}$ \textbf{の値がただ一つに定まる}(\textbf{ア},\textbf{ウ}は正しい)。\\
    \mit 2つの関数から,$z$ は $x$ の関数である(\textbf{イ}は正しい)。
\end{tabularx}\renewcommand{\arraystretch}{1}}
 & \cellcolor{mygray}
\\\hline
\scriptsize(4) & {\cellcolor{c041}\scriptsize
\renewcommand{\arraystretch}{1.6}
\begin{tabularx}{\linewidth}{X}
    \mit $x$ に $(t+2)$ を代入して,式を整理する。
\end{tabularx}\renewcommand{\arraystretch}{1}}
 & \cellcolor{mygray}
 & \cellcolor{mygray}
\\\hline
\scriptsize(5) & \cellcolor{mygray}
 & \cellcolor{mygray}
 & {\cellcolor{c053}\scriptsize
\renewcommand{\arraystretch}{1.6}
\begin{tabularx}{\linewidth}{X}
    \mit \textbf{切片}とは,\textbf{直線と} $\bm{y}$ \textbf{軸との交点}である。
\end{tabularx}\renewcommand{\arraystretch}{1}}
\\\hline
\scriptsize(6) & \cellcolor{mygray}
 & \cellcolor{mygray}
 & {\cellcolor{c063}\scriptsize
\renewcommand{\arraystretch}{1.6}
\begin{tabularx}{\linewidth}{X}
    \mit 2点 $(-3,\ 0)$,$(0,\ -1)$ を読み取る。\\
    \mit $\text{傾き}=\text{2点間の変化の割合}=\dfrac{\bm{y}\text{\textbf{の増加量}}}{\bm{x}\text{\textbf{の増加量}}}$\\
    \mit グラフに示された2点間の,$x$ の増加量は $3$,$y$ の増加量は $-1$ である。
\end{tabularx}\renewcommand{\arraystretch}{1}}
\\\hline
\scriptsize(7) & \cellcolor{mygray}
 & {\cellcolor{c072}\scriptsize
\renewcommand{\arraystretch}{1.6}
\begin{tabularx}{\linewidth}{X}
    \mit $\bm{y}$ \textbf{が} $\bm{x}$ \textbf{の関数である}とき,$\bm{x}$ \textbf{の値を決めると} $\bm{y}$ \textbf{の値がただ一つに定まる}。\\
    \mit 関数の定義をグラフと結びつけて捉える(右記第3項目)。
\end{tabularx}\renewcommand{\arraystretch}{1}}
 & {\cellcolor{c073}\scriptsize
\renewcommand{\arraystretch}{1.6}
\begin{tabularx}{\linewidth}{X}
    \mit グラフが\textbf{曲線}のとき,\textbf{変化の割合は一定でない}(\textbf{ア}は誤り)。\\
    \mit 点 $(0,\ 2)$ は,$x=0$ のとき $y=2$ であることを表す(\textbf{ウ}は誤り)。\\
    \mit $y=2$ となる $x$ の値が3つあることを読み取る など(\textbf{エ}は誤り)。
\end{tabularx}\renewcommand{\arraystretch}{1}}
\\\hline
\scriptsize(8) & \cellcolor{mygray}
 & \cellcolor{mygray}
 & {\cellcolor{c083}\scriptsize
\renewcommand{\arraystretch}{1.6}
\begin{tabularx}{\linewidth}{X}
    \mit 線分 $\mathrm{AB}$ は $y$ 軸に平行である。\\
    \mit 2点 $(3,\ 1)$,$(3,\ -2)$ の間に格子を意識して数える,または $y$ 座標の差 $1-(-2)$ を計算する。
\end{tabularx}\renewcommand{\arraystretch}{1}}
\\\hline
\scriptsize(9) & {\cellcolor{c091}\scriptsize
\renewcommand{\arraystretch}{1.6}
\begin{tabularx}{\linewidth}{X}
    \mit 原点 $\mathrm{O}$ と点 $(4,\ 4)$ を通る直線の式を求める($y=ax$ へ代入する場合)。\\
    \mit 2点 $(0,\ 2)$,$(4,\ 0)$ を通る直線の式を求める($y=ax+b$ へ代入する場合)。\\
    \mit 2つの直線の式 $y=x$,$y=-\dfrac{1}{2}x+2$ の連立方程式を解く。
\end{tabularx}\renewcommand{\arraystretch}{1}}
 & \cellcolor{mygray}
 & {\cellcolor{c093}\scriptsize
\renewcommand{\arraystretch}{1.6}
\begin{tabularx}{\linewidth}{X}
    \mit 原点 $\mathrm{O}$ と $(4,\ 4)$ を通る直線は,比例 $y=x$ のグラフである。\\
    \mit 2点 $(0,\ 2)$,$(4,\ 0)$ を通る直線の式を求める(切片と傾きをグラフから読み取る場合)。\\
    \mit 2直線の\textbf{交点の座標}は,2つの直線の式の\textbf{連立方程式の解}である。
\end{tabularx}\renewcommand{\arraystretch}{1}}
\\\hline
\scriptsize(10) & {\cellcolor{c101}\scriptsize
\renewcommand{\arraystretch}{1.6}
\begin{tabularx}{\linewidth}{X}
    \mit 方程式 $3t-(t-2) = 7$ を立て,解く。
\end{tabularx}\renewcommand{\arraystretch}{1}}
 & \cellcolor{mygray}
 & {\cellcolor{c103}\scriptsize
\renewcommand{\arraystretch}{1.6}
\begin{tabularx}{\linewidth}{X}
    \mit 点 $\mathrm{A}$ の座標を $(t,\ 3t)$,点 $\mathrm{B}$ の座標を $(t,\ t-2)$ と,$t$ を用いて表す。\\
    \mit 線分 $\mathrm{AB}$ の長さを,$3t-(t-2)$ と表す。
\end{tabularx}\renewcommand{\arraystretch}{1}}
\\\hline
\scriptsize(11) & \cellcolor{mygray}
 & {\cellcolor{c112}\scriptsize
\renewcommand{\arraystretch}{1.6}
\begin{tabularx}{\linewidth}{X}
    \mit 関数の定義を具体例と結びつけて捉える。\\
    \mit 関数の定義をもとに,\textbf{一方の値を決めたときに他方の値が決まるかどうか}を考える。\\
    \mit 「最初に水槽に入っていた水の高さ」がどうであっても,「蛇口から1秒あたりに出る水の容積」は決まらない(関数関係にない)。\\
    \mit 「最初に水槽に入っていた水の高さ」が高くなれば,その分「水を加えた後の水槽内の水の高さ」は高くなる(関数関係にある)。\\
    \mit 「蛇口から1秒あたりに出る水の容積」が大きくなれば,その分「水を加えた後の水槽内の水の高さ」は高くなる(関数関係にある)。\\
    \mit 「蛇口から1秒あたりに出る水の容積」が大きくなれば,その分「水を注ぐ時間」は短くて済む(関数関係にある)。
\end{tabularx}\renewcommand{\arraystretch}{1}}
 & \cellcolor{mygray}
\\\hline
\scriptsize(12) & {\cellcolor{c121}\scriptsize
\renewcommand{\arraystretch}{1.6}
\begin{tabularx}{\linewidth}{X}
    \mit 必要な変数を,適切に文字 $x$,$y$ に設定する。\\
    \mit 設定した変数に合うような式を立てる。\\
\end{tabularx}\renewcommand{\arraystretch}{1}}
 & {\cellcolor{c122}\scriptsize
\renewcommand{\arraystretch}{1.6}
\begin{tabularx}{\linewidth}{X}
    \mit 関数の定義を具体例と結びつけて捉える。\\
    \mit (11)と同様に,関数関係にある2つの数量を見つける。
\end{tabularx}\renewcommand{\arraystretch}{1}}
 & \cellcolor{mygray}
\\\hline
\scriptsize(13) & \cellcolor{mygray}
 & \cellcolor{mygray}
 & {\cellcolor{c133}\scriptsize
\renewcommand{\arraystretch}{1.6}
\begin{tabularx}{\linewidth}{X}
    \mit \textbf{傾き} $\bm{a}$ \textbf{が} $\mathbf{-1}<\bm{a}<\mathbf{0}$ のとき,傾き $-1$ のグラフよりも,\textbf{傾きが緩やか}である。\\
    \mit \textbf{傾き} $\bm{a}$ \textbf{が負}のとき,\textbf{グラフは右下がり}になる。
\end{tabularx}\renewcommand{\arraystretch}{1}}
\\\hline
\scriptsize(14) & \cellcolor{mygray}
 & {\cellcolor{c142}\scriptsize
\renewcommand{\arraystretch}{1.6}
\begin{tabularx}{\linewidth}{X}
    \mit $\bm{y}$ \textbf{が} $\bm{x}$ \textbf{の関数である}とき,$\bm{x}$ \textbf{の値を決めると} $\bm{y}$ \textbf{の値がただ一つに定まる}。\\
    \mit 関数の定義をグラフと結びつけて捉える。\\
    \mit 関数は,比例,反比例,一次関数,関数 $y=ax^2$ だけに限られない。
\end{tabularx}\renewcommand{\arraystretch}{1}}
 & {\cellcolor{c143}\scriptsize
\renewcommand{\arraystretch}{1.6}
\begin{tabularx}{\linewidth}{X}
    \mit \textbf{ア}は,一つの $x$ の値に対して二つの $y$ の値が対応していることがあることを読み取る。(「$y$ は $x$ の関数である」とはいえない。)\\
    \mit \textbf{イ},\textbf{ウ},\textbf{エ}は,$x$ の値を決めると,$y$ の値がただ一つに定まっている。
\end{tabularx}\renewcommand{\arraystretch}{1}}
\\\hline
\end{tabularx}