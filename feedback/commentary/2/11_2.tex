\renewcommand{\arraystretch}{1.6}
\begin{tabularx}{\linewidth}{X}
    \mit 関数の定義を具体例と結びつけて捉える。\\
    \mit 関数の定義をもとに,\textbf{一方の値を決めたときに他方の値が決まるかどうか}を考える。\\
    \mit 「ボールペンAで引いた線の長さ」が長くなれば,「ボールペンAのインクの使用量」は増える(関数関係にある)。\\
    \mit \textbf{イ}はボールぺンBに関して同上。\\
    \mit 「ボールペンAのボール径」が大きくなれば,「ボールペンAのインクの使用量」は増える(関数関係にある)。\\
    \mit 「ボールペンAのボール径」を決めても,「ボールペンBのインクの使用量」は決まらない(関数関係にない)。\\
\end{tabularx}\renewcommand{\arraystretch}{1}