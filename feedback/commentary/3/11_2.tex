\renewcommand{\arraystretch}{1.6}
\begin{tabularx}{\linewidth}{X}
    \mit 関数の定義を具体例と結びつけて捉える。\\
    \mit 関数の定義をもとに,\textbf{一方の値を決めたときに他方の値が決まるかどうか}を考える。\\
    \mit 「Aさんの歩く速さ」が速くなれば,「CさんがAさんに追いつく時刻」は遅くなる(関数関係にある)。\\
    \mit 「Cさんが家を出る時間」が遅くなれば,「CさんがAさんに追いつく時刻」は遅くなる(関数関係にある)。\\
    \mit 「Bさんの歩く速さ」が速くなっても,「CさんがAさんに追いつく時刻」には影響しない(関数関係にない)。\\
    \mit 「Cさんの走る速さ」が速くなれば,「CさんがAさんに追いつく時刻」は速くなる(関数関係にある)。\\
\end{tabularx}\renewcommand{\arraystretch}{1}