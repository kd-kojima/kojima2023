\renewcommand{\arraystretch}{1.6}
\begin{tabularx}{\linewidth}{X}
    \mit 関数の定義を具体例と結びつけて捉える。\\
    \mit 関数の定義をもとに,\textbf{一方の値を決めたときに他方の値が決まるかどうか}を考える。\\
    \mit 「最初に水槽に入っていた水の高さ」がどうであっても,「蛇口から1秒あたりに出る水の容積」は決まらない(関数関係にない)。\\
    \mit 「最初に水槽に入っていた水の高さ」が高くなれば,その分「水を加えた後の水槽内の水の高さ」は高くなる(関数関係にある)。\\
    \mit 「蛇口から1秒あたりに出る水の容積」が大きくなれば,その分「水を加えた後の水槽内の水の高さ」は高くなる(関数関係にある)。\\
    \mit 「蛇口から1秒あたりに出る水の容積」が大きくなれば,その分「水を注ぐ時間」は短くて済む(関数関係にある)。
\end{tabularx}\renewcommand{\arraystretch}{1}