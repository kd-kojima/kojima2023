\textbf{測定能力別 達成状況}\par
\begin{figure}[h]
    \begin{tikzpicture}
        \draw (-1,3) node[left]{\textbf{\small 代数的手続きの理解}} node at (-5,3){\textbf{\small 1}};
        \draw (-1,1.9) node[left]{\textbf{\small 関数的思考の理解}} node at (-5,1.9){\textbf{\small 2}};
        \draw (-1,0.8) node[left]{\textbf{\small 直交座標上での表現の理解}} node at (-5,0.8){\textbf{\small 3}};
        \draw (0,3.1) node[left]{\scriptsize 第\!\ 2\!\ 回};
        \draw (0,2.7) node[left]{\tiny 第\!\ 1\!\ 回};
        \draw (0,2) node[left]{\scriptsize 第\!\ 2\!\ 回};
        \draw (0,1.6) node[left]{\tiny 第\!\ 1\!\ 回};
        \draw (0,0.9) node[left]{\scriptsize 第\!\ 2\!\ 回};
        \draw (0,0.5) node[left]{\tiny 第\!\ 1\!\ 回};

        % <here>
        % \fill[gray!100!red] (0,3.4) rectangle (8,2.8);
        % \fill[lightgray] (0,2.6) rectangle (4,2.8);
        % \fill[gray!90!red] (0,2.3) rectangle (7,1.7);
        % \fill[lightgray] (0,1.5) rectangle (4,1.7);
        % \fill[gray!80!red] (0,1.2) rectangle (6,0.6);
        % \fill[lightgray] (0,0.4) rectangle (4,0.6);

        \draw[thick] (0,0)--(0,4);
        \draw (0,0)--(10.7,0);
        \draw[thick] (0,0)--(0,-0.1) node[below]{\scriptsize 0};
        \foreach \x [count=\i] in {1,...,10} {
            \draw (\x,0)--(\x,-0.1) node[below]{\scriptsize \i0};
        }
        \draw (11.5, -0.7) node[left]{\scriptsize ※第\!\ 1\!\ 回の分析の際に少しミスがありましたので,本フィードバックシートの第\!\ 1\!\ 回の達成状況と,前回のフィードバックシートの達成状況が一部異なる場合があります。};
    \end{tikzpicture}
\end{figure}\par