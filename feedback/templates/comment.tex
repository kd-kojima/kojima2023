\textbf{測定能力別 コメント}\par\vspace*{0.3\baselineskip}
\begin{tabularx}{0.98\linewidth}{clX}
    \textbf{1} & \underline{\textbf{代数的手続きの理解}} & 文字を使った式を扱うことができる。それぞれの関数の一般形($y=ax+b$ など)を扱うことができる。自力で文字を使って式を立てられる。\\
\end{tabularx}\par\vspace*{-0.8em}
\hspace*{3em}\begin{screen}
    % <here>
    % あなたの達成状況は\textbf{46}\%です。あなたは文字を使うことに慣れてきているようです。まだうまく活用できない場合もあるようですが、この調子で勉強を進め,文字の扱いに慣れていくことができれば、「代数的手続き」の理解度は高まっていくでしょう。
\end{screen}\par\vspace*{0.5\baselineskip}
\begin{tabularx}{0.98\linewidth}{clXX}
    \textbf{2} & \underline{\textbf{関数的思考の理解}} & 関数の定義「一方の値を決めると他方の値がただ一つに定まる」を理解できる。具体的な例を関数として捉えることができる。\\
\end{tabularx}\par\vspace*{-0.8em}
\hspace*{3em}\begin{screen}
    % <here>
    % あなたの達成状況は\textbf{46}\%です。あなたは文字を使うことに慣れてきているようです。まだうまく活用できない場合もあるようですが、この調子で勉強を進め,文字の扱いに慣れていくことができれば、「代数的手続き」の理解度は高まっていくでしょう。
\end{screen}\par\vspace*{0.5\baselineskip}
\begin{tabularx}{0.98\linewidth}{clXX}
    \textbf{3} & \underline{\textbf{直交座標上での表現の理解}} & 直交座標上にかかれたグラフが意味する内容や,グラフの特徴が意味する内容を,数学的用語や数式と結び付けて理解できる。\\
\end{tabularx}\par\vspace*{-0.8em}
\hspace*{3em}\begin{screen}
    % <here>
    % あなたの達成状況は\textbf{46}\%です。あなたは文字を使うことに慣れてきているようです。まだうまく活用できない場合もあるようですが、この調子で勉強を進め,文字の扱いに慣れていくことができれば、「代数的手続き」の理解度は高まっていくでしょう。
\end{screen}\vspace*{0.5\baselineskip}
