\documentclass[uplatex, dvipdfmx, a4paper, 10pt]{jsarticle}
\usepackage{graphicx}
\usepackage[T1]{fontenc}
\usepackage{amsmath,amssymb, amsthm, mathtools}
\usepackage[uplatex, deluxe, bold]{otf}

\usepackage[noalphabet, unicode]{pxchfon}
\setminchofont[0]{YuMincho.ttc}
\setgothicfont[0]{YuGo-Medium.otf}
\setboldgothicfont[0]{YuGo-Bold.otf}
\setxboldgothicfont[0]{YuGo-Bold.otf}

\usepackage{tikz}
\usetikzlibrary{intersections, calc, arrows.meta, angles, quotes, shapes.callouts}
\usepackage{pgfplots}
\pgfplotsset{compat=newest}
\usepackage{fancyhdr}
\usepackage{newtxmath}
\let\widering\relax
\usepackage{fouriernc}
% \usepackage[scale=1]{tgschola}
\usepackage[e]{esvect}
\usepackage{esint}
\usepackage{tabularx, booktabs}
\usepackage{dcolumn}
\usepackage{bm}

\usepackage{pxrubrica}
\kentenmarkintate{bullet}

\setlength{\textheight}{\paperheight}
\setlength{\topmargin}{20truemm}
\addtolength{\textheight}{-\topmargin}
\addtolength{\textheight}{-\topmargin}
\addtolength{\topmargin}{-1truein}
\addtolength{\topmargin}{-\headheight}
\addtolength{\topmargin}{-\headsep}

\setlength{\textwidth}{\paperwidth}
\setlength{\oddsidemargin}{20truemm}
\addtolength{\textwidth}{-\oddsidemargin}
\addtolength{\textwidth}{-\oddsidemargin}
\addtolength{\oddsidemargin}{-1truein}
\setlength{\evensidemargin}{\oddsidemargin}

% \newcolumntype{C}{>{\centering\arraybackslash}X}
% \newcolumntype{L}{>{\raggedright\arraybackslash}X}
% \newcolumntype{R}{>{\raggedleft\arraybackslash}X}

\renewcommand{\sfdefault}{phv}

\renewcommand{\gt}[1]{\textgt{\textbf{#1}}}
\renewcommand{\sf}[1]{\textsf{#1}}

% \renewcommand\UrlFont{\rmfamily}
\newcommand{\vs}{\vspace{\baselineskip}}
\renewcommand{\t}[1]{(#1)\hspace*{1em}}
\newcommand{\mvv}[1]{\vv{\mathstrut #1}}
\newcommand{\mbar}[1]{\overline{\mathstrut #1}}
\newcommand{\mabs}[1]{\raisebox{-1.2pt}{\scalebox{0.6}[1.4]{|}}#1\raisebox{-1.2pt}{\scalebox{0.6}[1.4]{|}}}
\newcommand{\abs}[1]{\left\lvert#1\right\rvert}
\newcommand{\mg}{\scalebox{1.13}[1]{$\varg$}\hspace*{1pt}}
\newcommand{\hang}[1]{\settowidth{\hangindent}{#1}#1}
\newcommand{\prll}{\:\,/\!\:\!/\:\,}
\renewcommand{\sim}{∽\hspace*{-0.01pt}\:}
\DeclareMathOperator*{\msum}{\text{\Large \raisebox{-2pt}{$\Sigma$}}}

\makeatletter
\DeclareFontFamily{U}{tipa}{}
\DeclareFontShape{U}{tipa}{m}{n}{<->tipa10}{}
\newcommand{\arc@char}{{\usefont{U}{tipa}{m}{n}\symbol{62}}}%
\newcommand{\arc}[1]{\mathpalette\arc@arc{#1}}
\newcommand{\arc@arc}[2]{%
  \sbox0{$\m@th#1#2$}%
  \vbox{
    \hbox{\resizebox{\wd0}{\height}{\arc@char}}
    \nointerlineskip
    \box0
  }%
}
\makeatother

\newcommand{\ctext}[1]{\raisebox{0.2ex}{\textcircled{\raisebox{-0.12ex}{\small{#1}}}}}

\newcommand{\toi}{\stepcounter{toicnt}\noindent(\thetoicnt)\hspace*{1em}\setcounter{subtoicnt}{0}}
\newcommand{\subtoi}{\stepcounter{subtoicnt}(\thesubtoicnt)\hspace*{1em}\setcounter{subsubtoicnt}{0}}
\newcommand{\subsubtoi}{\stepcounter{subsubtoicnt}\ctext{\thesubsubtoicnt}\hspace*{1em}}


\newcommand{\romani}{$\rm \hspace{.18em}i\hspace{.18em}$}
\newcommand{\romanii}{$\rm \hspace{.08em}ii\hspace{.08em}$}
\newcommand{\romaniii}{$\rm i\hspace{-.08em}i\hspace{-.08em}i$}
\newcommand{\romaniv}{$\rm i\hspace{-.08em}v\hspace{-.06em}$}
\newcommand{\romanv}{$\rm \hspace{.06em}v\hspace{.06em}$}
\newcommand{\romanvi}{$\rm \hspace{-.06em}v\hspace{-.08em}i$}
\newcommand{\romanI}{$\rm \:\,I\:\,$}
\newcommand{\romanII}{$\rm \;\!I\hspace{-.08em}I\;\!$}
\newcommand{\romanIII}{$\rm I\hspace{-.15em}I\hspace{-.15em}I$}

\newcommand{\atoe}{\gt{ア}~\gt{エ}}
\newcommand{\atoo}{\gt{ア}~\gt{オ}}
\newcommand{\atoka}{\gt{ア}~\gt{カ}}

\newcommand{\choicefour}[4]{%
    \noindent\hskip2zw\gt{ア}\hskip1zw #1\par%
    \noindent\hskip2zw\gt{イ}\hskip1zw #2\par%
    \noindent\hskip2zw\gt{ウ}\hskip1zw #3\par%
    \noindent\hskip2zw\gt{エ}\hskip1zw #4\par%
}
\newcommand{\choicefive}[5]{%
    \noindent\hskip2zw\gt{ア}\hskip1zw #1\par%
    \noindent\hskip2zw\gt{イ}\hskip1zw #2\par%
    \noindent\hskip2zw\gt{ウ}\hskip1zw #3\par%
    \noindent\hskip2zw\gt{エ}\hskip1zw #4\par%
    \noindent\hskip2zw\gt{オ}\hskip1zw #5\par%
}
\newcommand{\choicesix}[6]{%
    \noindent\hskip2zw\gt{ア}\hskip1zw #1\par%
    \noindent\hskip2zw\gt{イ}\hskip1zw #2\par%
    \noindent\hskip2zw\gt{ウ}\hskip1zw #3\par%
    \noindent\hskip2zw\gt{エ}\hskip1zw #4\par%
    \noindent\hskip2zw\gt{オ}\hskip1zw #5\par%
    \noindent\hskip2zw\gt{カ}\hskip1zw #6\par%
}

\newcommand{\attr}[4][]{%
    \vs\vs
    % {\scriptsize\begin{tabular}[h]{|c|c|c|l|}
    %     \hline
    %     A1 & A2 & A3 & 備考\\\hline
    %     #2 & #3 & #4 & #1\\\hline
    % \end{tabular}}\par
}
\newcommand{\ans}{\par\vs\vs\noindent\rightline{\underline{(答\thetoicnt)\hspace*{15zw}}}\par}

\setlength\normallineskiplimit{6pt}
\setlength\normallineskip{6pt}

% \renewcommand{\headrulewidth}{0pt}
\renewcommand{\footrulewidth}{1pt}

\begin{document}

\newcounter{toicnt}
\newcounter{subtoicnt}
\newcounter{subsubtoicnt}
\setcounter{toicnt}{0}

\pagestyle{fancy}
\lhead{}
\cfoot{\thepage}
\vspace*{10\baselineskip}
\begin{center}
    \noindent{\LARGE\gt{関  数}}\par\vs
    \noindent{\Large\gt{学習状況診断テスト}}\par \vs
    \noindent{\Large\gt{第 \sf{3} 回}}\par \vs
    \vspace*{3\baselineskip}
    \noindent\begin{tabularx}{0.8\linewidth}{X}
        \toprule
        このテストは,「関数」に関する学習状況を診断して,今後の勉強に活かすためのものです。このテストの結果が成績や評価に関係することはありませんが,今後の勉強に役立つアドバイスがもらえるよう,真剣に取り組んでください。\\
        問題は全部で14問あります。テストは20分で取り組んでください。わからない問題は後回しにして,\underline{最後の問題まで考えるようにしましょう。}\\
        \toprule
    \end{tabularx}\par 
    \vspace*{10\baselineskip}
    \noindent\underline{\Large\gt{生徒個人番号}\hspace*{20zw}}
\end{center}

\newpage
\vspace*{10zw}
\newpage
% \parindent = 0pt
\lhead{関数 $-$ 学習状況診断テスト}

%
% === 1 ====
%
\attr{1}{0}{0}
\toi 一次関数 $y=3x+2$ において,$y=4$ であるとき,$x$ を求めてください。\par\vs\vs\vs
\ans
\par\vs 

%
% === 2 ===
%
\attr{1}{0}{0}
\toi 一次関数において,$x=-2$ のとき $y=2$,$x=3$ のとき $y=12$ である。このとき,$y$ を $x$ の式で表してください。\par\vs\vs\vs
\ans
\par\vs 

%
% === 3 ===
%
\attr{0}{1}{0}
\toi 3つの数量 $a$,$b$,$c$ がある。$b$ は $a$ の関数であり,$c$ は $b$ の関数である。この2つの関数について,関数の定義をもとに,\gt{\kenten{必ずしも正しいといえないもの}}を次の\atoe のうちから\gt{\kenten{一つ}}選んでください。\par\vs
\choicefour{
    $c$ の値を決めると,$b$ の値がただ一つに定まる。}{
    $a$ の値を決めると,$c$ の値がただ一つに定まる。}{
    $b$ の値を決めると,$c$ の値がただ一つに定まる。}{
    $a$ の値を決めると,$b$ の値がただ一つに定まる。
}
\ans
\par\vs 

%
% === 4 ===
%
\attr{1}{0}{0}
\toi 一次関数 $y=3x-2$ において,$y$ が $t$ であるとき,$x$ を $t$ を用いて表してください。\par\vs\vs\vs
\ans
\par\newpage

%
% === 5 ===
%
\attr{0}{0}{1}
\noindent
\begin{minipage}{0.55\linewidth}
    \toi 右の図に,一次関数のグラフがかかれている。この一次関数の\gt{\kenten{切片}}を答えてください。\par\vs\vs\vs
    \ans
\end{minipage}\hspace*{4zw}
\begin{minipage}{0.4\linewidth}
    \begin{tikzpicture}[x=5mm, y=5mm]
        \draw (-5,0)--(5,0) node[right]{$x$}; \draw (0,-6)--(0,4) node[above]{$y$}; \draw (0,0) node[below right]{O};
        \draw[semithick, samples=100, domain=-3:2] plot(\x, -2*\x - 2);
        \draw (0,-2) node[right]{$-2$};
        \draw (-1.3,0) node[below]{$-1$};
    \end{tikzpicture}
\end{minipage}
\par\vs 

%
% === 6 ===
%
\attr{0}{0}{1}
\noindent
\begin{minipage}{0.55\linewidth}
    \toi 右の図に,一次関数のグラフがかかれている。この一次関数の\gt{\kenten{傾き}}を答えてください。\par\vs\vs\vs
    \ans
\end{minipage}\hspace*{4zw}
\begin{minipage}{0.4\linewidth}
    \begin{tikzpicture}[x=5mm, y=5mm]
        \draw (-5,0)--(5,0) node[right]{$x$}; \draw (0,-4)--(0,6) node[above]{$y$}; \draw (0,0) node[below left]{O};
        \draw[semithick, samples=100, domain=-5:1.5] plot(\x, 4*\x/3 + 4);
        \draw (0,4) node[right]{$4$};
        \draw (-2.8,0) node[below]{$-3$};
    \end{tikzpicture}
\end{minipage}
\par\vs 

%
% === 7 ===
%
\attr{0}{1}{1}
\noindent
\begin{minipage}{0.6\linewidth}
    \toi 右の図は,ある関数のグラフをかいたものである。この関数について,\gt{\kenten{誤っているもの}}を次の\atoe のうちから\gt{\kenten{一つ}}選んでください。\par\vs

    \choicefour{
        $x=-2$ のとき,$y=-1$ である。}{
        $x$ の値を一つ決めると,$y$ の値もただ一つに定まる。}{    
        $x$ の値が増加するとき,$y$ の値も必ず増加する。}{
        $0 \leqq x \leqq 2$ の範囲では,$y$ の値は必ず正である。
    }
    \ans
\end{minipage}\hspace*{2.5zw}
\begin{minipage}{0.35\linewidth}
    \begin{tikzpicture}[x=6.5mm, y=6.5mm]
        \draw (-4,0)--(4,0) node[right]{$x$}; \draw (0,-3)--(0,5) node[above]{$y$}; \draw (0,0) node[below right]{O};
        \draw[semithick, samples=100, domain=-3.52:3.52] plot(\x, -\x*\x*\x/4 + 2*\x + 1);
        \draw (0,1) node[right]{$1$}; \draw (-2,0) node[above]{$-2$}; \draw (0,-1) node[right]{$-1$};
        \draw (2,0) node[below]{$2$}; \draw (0,3) node[left]{$3$};
        \fill (-2,-1) circle (2pt); \fill (2,3) circle (2pt);
        \draw[dashed] (-2,-1)--(-2,0); \draw[dashed] (-2,-1)--(0,-1); \draw[dashed] (2,3)--(2,0); \draw[dashed] (2,3)--(0,3);
    \end{tikzpicture}
\end{minipage}
\par\vs\vs
\par\newpage 

%
% === 8 ===
%
\attr{0}{0}{1}
\noindent
\begin{minipage}{0.55\linewidth}
    \toi 右の図には,反比例 $y=\dfrac{3}{x}$ のグラフと,比例 $y=-2x-2$ のグラフがかかれている。$y=\dfrac{3}{x}$ 上に点 $\mathrm{A(3,\ 1)}$,$y=-2x-2$ 上に点 $\mathrm{B\left(-\dfrac{3}{2},\ 1\right)}$ をとるとき,線分ABの長さを求めてください。\par 
    ただし,原点Oから点 $(0,\ 1)$ および原点Oから点 $(1,\ 0)$ の長さを1cmとする。\par\vs\vs
    \ans
\end{minipage}\hspace*{4zw}
\begin{minipage}{0.4\linewidth}
    \begin{tikzpicture}[x=6mm, y=6mm]
        \draw (-4,0)--(5,0) node[right]{$x$}; \draw (0,-3)--(0,6) node[above]{$y$}; \draw (0,0) node[below right]{O};
        \draw[semithick, domain=-3.8:0.5] plot(\x, -2*\x - 2);
        \draw[semithick, samples=100, domain=-4:-1] plot(\x, 3/\x);
        \draw[semithick, samples=100, domain=0.5:5] plot(\x, 3/\x);
        \fill (3,1) circle (2pt) node at (3,1)[above]{A};
        \fill (-1.5,1) circle (2pt) node at (-1.5,1)[left]{B};
    \end{tikzpicture}
\end{minipage}
\par\vs

%
% === 9 ===
%
\attr{1}{0}{1}
\begin{minipage}{0.55\linewidth}
    \toi 右の図の2つのグラフの交点の座標を求めてください。\par\vs\vs\vs\vs\vs\vs\vs
    \rightline{\underline{(答\thetoicnt)\hspace*{1zw}{\LARGE(\hspace*{4zw},\hspace*{4zw})}\hspace*{2zw}}}\par
\end{minipage}\hspace*{4zw}
\begin{minipage}{0.4\linewidth}
    \begin{tikzpicture}[x=6.5mm, y=6.5mm]
        \draw (-2,0)--(6,0) node[right]{$x$}; \draw (0,-5)--(0,3) node[above]{$y$}; \draw (0,0) node[below right]{O};
        \draw[semithick, samples=100, domain=-1.8:5.8] plot(\x, \x/3);
        \draw[semithick, samples=100, domain=-0.7:5.2] plot(\x, 4*\x/3-4);
        \draw (0,-4) node[right]{$-4$}; \draw (3.1,0) node[below]{3};
        \fill (3,1) circle (2pt) node at (3,1)[above]{$(3,\ 1)$};
    \end{tikzpicture}
\end{minipage}\par

%
% === 10 ===
%
\attr{1}{0}{1}
\noindent\begin{minipage}{0.5\linewidth}
    \toi 右の図のように,直線 $y=\dfrac{4}{3}x+1$ と直線 $y=\dfrac{1}{2}x$ がある。$x$ 軸上に点Pをとり,Pを通り $y$ 軸に平行な直線と,直線 $y=\dfrac{4}{3}x+1$,直線 $y=\dfrac{1}{2}x$ との交点をそれぞれ点A,Bとする。線分ABの長さが $3$ となるときの点Pの $x$ 座標を求めてください。\par 
    ただし,点Pの $x$ 座標を $t$ とおき,ABの長さに関する方程式を立てることによって求めること。\par 
    また,点Pの $x$ 座標は正とする。\par \vs\vs
\end{minipage}\hspace*{3zw}
\begin{minipage}{0.45\linewidth}
    \begin{tikzpicture}[x=4mm, y=4mm]
        \draw (-4,0)--(12,0) node[right]{$x$}; \draw (0,-4)--(0,12) node[above]{$y$}; \draw (0,0) node[below right]{O};
        \draw[semithick, samples=100, domain=-3.7:8.2] plot(\x, 4*\x/3+1) node[above right]{$y=\dfrac{4}{3}x+1$};
        \draw[semithick, samples=100, domain=-4:12] plot(\x, \x/2) node[below]{$y=\dfrac{1}{2}x$};
        \draw (3,11)--(3,-3);
        \draw (3,0) node[below right]{$\mathrm{P}(t,\ 0)$}; \draw (3,1.8) node[below right]{B}; \draw (3,5) node[right]{A};
    \end{tikzpicture}
\end{minipage}\par\vs\vs\vs\ans

%
% === 11 ===
%
\attr{0}{1}{0}
\toi 次の\gt{文章}について,正しい記述を\atoe のうちから\gt{\kenten{一つ}}選んでください。\par\vs
\noindent\gt{文章}\hspace*{1zw}
\fbox{\parbox{0.92\linewidth}{
    AさんとBさんが15時00分に家を出ました。Aさんは分速60mで,Bさんは分速70mで歩いています。Aさんの忘れ物に気づいたCさんは,15時04分に家を出て分速100mで走ってAさんを追いかけました。CさんはAさんに15時10分に追いつきました。
}}\par\vs
\choicefour{
    「Aさんの歩く速さ」と「CさんがAさんに追いつく時刻」は\gt{関数関係にない}。}{
    「CさんがAさんに追いつく時刻」と「Cさんが家を出る時刻」は\gt{関数関係にない}。}{
    「Bさんの歩く速さ」と「CさんがAさんに追いつく時刻」は\gt{関数関係にない}。}{
    「CさんがAさんに追いつく時刻」と「Cさんの走る速さ」は\gt{関数関係にない}。
}
\ans\par
\par 

%
% === 12 ===
%
\attr{1}{1}{0}
\toi 次の\gt{文章}中の2つの数量を選び,関数関係を式に表してください。\par 
ただし,\gt{例}を参考に,2つの数量は\atoe のうちから記号で選び,どの数量を文字 $x$,$y$ としたか示したうえで,$y$ を $x$ の式で表してください。\par 
また,正答は複数個あり,そのうちのいずれを答えても正解です。\par\vs
\gt{例}\hspace*{1zw}
\fbox{\parbox{0.92\linewidth}{
    \gt{文章}\hspace*{1zw}\fbox{Aさんは分速80mで5分歩くと,400m進みました。}\\
    \hspace*{2zw}\gt{ア} 歩いた速さ\\
    \hspace*{2zw}\gt{イ} 歩いた時間\\
    \hspace*{2zw}\gt{ウ} 歩いた距離\\\hrulefill\\
    \gt{解答}\hspace*{2zw}\underline{$x$:\gt{イ},$y$:\gt{ウ},関係式:$y = 80x$}\\
    \hspace*{4zw}※ \underline{$x$:\gt{イ},$y$:\gt{ア},関係式:$y = \dfrac{400}{x}$} や \underline{$x$:\gt{ア},$y$:\gt{ウ},関係式:$y = 5x$} なども正解です。
}}\par\vs
\noindent\gt{文章}\hspace*{1zw}\fbox{\parbox{0.92\linewidth}{
    歯数15の歯車Sと歯数30の歯車Tがかみ合って回転するとき,歯車Tが1回転すると歯車Sは2回転します。また,歯車Sと歯数10の歯車Uがかみ合って回転するとき,歯車Uが1回転すると歯車Sは $\dfrac{2}{3}$ 回転します。
}}\par\vs
\choicefour{歯車Sの歯数}{歯車Tの回転数}{歯車Uの回転数}{歯車Uの歯数}\par\vs\vs
\rightline{\underline{(答\thetoicnt) $x$:\hspace*{7zw} $y$:\hspace*{7zw}関係式:\hspace*{15zw}}}\par\vspace*{.5em}
\newpage

%
% === 13 ===
%
\attr{0}{0}{1}
\toi 傾き $a$ を $a > 1$ の範囲で決定したとき,一次関数 $y = ax - 1$ のグラフはどのようにかけるか。次の\atoe のうちから\gt{\kenten{一つ}}選んでください。\par 
ただし,点線は $y=-x - 1$ および $y=x-1$ のグラフである。
\begin{table}[h]
    \begin{tabular}{|c|c||c|c|}
        \hline
        \begin{tikzpicture}[x=5mm, y=5mm]\draw(0,5)node{};\draw(0,-5)node{};\draw(0,0)node{\gt{ア}};\end{tikzpicture} &
        \begin{tikzpicture}[x=5mm, y=5mm]
            \draw (-5,0)--(5,0) node[right]{$x$}; \draw (0,-5)--(0,5) node[above]{$y$}; \draw (0,0) node[above left]{O};
            \draw[samples=100, domain=-4.7:3.7, dashed] plot(\x, -\x-1);
            \draw[samples=100, domain=-3.7:4.7, dashed] plot(\x, \x-1);
            \draw[semithick, samples=100, domain=-5:5] plot(\x, -\x/2-1);
        \end{tikzpicture} &
        \begin{tikzpicture}[x=5mm, y=5mm]\draw(0,5)node{};\draw(0,-5)node{};\draw(0,0)node{\gt{イ}};\end{tikzpicture} &
        \begin{tikzpicture}[x=5mm, y=5mm]
            \draw (-5,0)--(5,0) node[right]{$x$}; \draw (0,-5)--(0,5) node[above]{$y$}; \draw (0,0) node[above left]{O};
            \draw[samples=100, domain=-4.7:3.7, dashed] plot(\x, -\x-1);
            \draw[samples=100, domain=-3.7:4.7, dashed] plot(\x, \x-1);
            \draw[semithick, samples=100, domain=-3:2] plot(\x, -2*\x-1);
        \end{tikzpicture} \\\hline
        \begin{tikzpicture}[x=5mm, y=5mm]\draw(0,5)node{};\draw(0,-5)node{};\draw(0,0)node{\gt{ウ}};\end{tikzpicture} &
        \begin{tikzpicture}[x=5mm, y=5mm]
            \draw (-5,0)--(5,0) node[right]{$x$}; \draw (0,-5)--(0,5) node[above]{$y$}; \draw (0,0) node[above left]{O};
            \draw[samples=100, domain=-4.7:3.7, dashed] plot(\x, -\x-1);
            \draw[samples=100, domain=-3.7:4.7, dashed] plot(\x, \x-1);
            \draw[semithick, samples=100, domain=-2:3] plot(\x, 2*\x-1);
        \end{tikzpicture} &
        \begin{tikzpicture}[x=5mm, y=5mm]\draw(0,5)node{};\draw(0,-5)node{};\draw(0,0)node{\gt{エ}};\end{tikzpicture} &
        \begin{tikzpicture}[x=5mm, y=5mm]
            \draw (-5,0)--(5,0) node[right]{$x$}; \draw (0,-5)--(0,5) node[above]{$y$}; \draw (0,0) node[above left]{O};
            \draw[samples=100, domain=-4.7:3.7, dashed] plot(\x, -\x-1);
            \draw[samples=100, domain=-3.7:4.7, dashed] plot(\x, \x-1);
            \draw[semithick, samples=100, domain=-5:5] plot(\x, \x/2-1);
        \end{tikzpicture} \\\hline
    \end{tabular}
\end{table}
\ans
\newpage

%
% === 14 ===
%
\attr{0}{1}{1}
\toi 次の\atoe のグラフのうち,関数の定義に照らして,「$y$ は $x$ の関数である」と\gt{\kenten{いえないもの}}を\gt{\kenten{一つ}}選んでください。\par
\begin{table}[h]
    \begin{tabular}{|c|c||c|c|}
        \hline
        \begin{tikzpicture}[x=8mm, y=8mm]\draw(0,3)node{};\draw(0,-3)node{};\draw(0,0)node{\gt{ア}};\end{tikzpicture} &
        \begin{tikzpicture}[x=8mm, y=8mm]
            \draw (-3,0)--(3,0) node[right]{$x$}; \draw (0,-3)--(0,3) node[above]{$y$}; \draw (0,0) node[below right]{O};
            \draw[semithick, samples=100, domain=-1.82:3] plot(\x, {4*log10(\x+2)});
        \end{tikzpicture} &
        \begin{tikzpicture}[x=8mm, y=8mm]\draw(0,3)node{};\draw(0,-3)node{};\draw(0,0)node{\gt{イ}};\end{tikzpicture} & 
        \begin{tikzpicture}[x=8mm, y=8mm]
            \draw (-3,0)--(3,0) node[right]{$x$}; \draw (0,-3)--(0,3) node[above]{$y$}; \draw (0,0) node[below right]{O};
            \draw[semithick, samples=100, domain=-3:0] plot(\x, -\x/2+1);
            \draw[semithick, samples=100, domain=0:3] plot(\x, {-(\x-1)*(\x-1)+2});
        \end{tikzpicture}\\ \hline
        \begin{tikzpicture}[x=8mm, y=8mm]\draw(0,3)node{};\draw(0,-3)node{};\draw(0,0)node{\gt{ウ}};\end{tikzpicture} &
        \begin{tikzpicture}[x=8mm, y=8mm]
            \draw (-3,0)--(3,0) node[right]{$x$}; \draw (0,-3)--(0,3) node[above]{$y$}; \draw (0,0) node[below right]{O};
            \draw[semithick, samples=100, domain=-2.8:2.87] plot(\x, {exp(\x)/2-\x-3});
        \end{tikzpicture} &
        \begin{tikzpicture}[x=8mm, y=8mm]\draw(0,3)node{};\draw(0,-3)node{};\draw(0,0)node{\gt{エ}};\end{tikzpicture} & 
        \begin{tikzpicture}[x=8mm, y=8mm]
            \draw (-3,0)--(3,0) node[right]{$x$}; \draw (0,-3)--(0,3) node[above]{$y$}; \draw (0,0) node[below right]{O};
            \draw[semithick, samples=100, domain=-1:1.1] plot(\x, {sqrt(\x+1)-1});
            \draw[semithick, samples=100, domain=-1:3] plot(\x, {-sqrt(\x+1)-1});
            \draw[semithick, samples=100, domain=-3:1.1] plot(\x, -\x/2+1);
        \end{tikzpicture}\\ \hline
    \end{tabular}
\end{table}
\ans


\end{document}